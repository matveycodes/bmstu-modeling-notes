\section{Лекция №7 (23.10.2023)}

\subsection{Метод Монте-Карло (статистических испытаний)}

На практике далеко не все процессы являются марковскими или сводящимися к марковским. В системах массового обслуживания поток заявок не всегда бывает пуассоновским (простейшим), реже наблюдается постоянное распределение времени обслуживания. Для произвольных потоков событий, переводящих систему из состояния в состояние, аналитические решения получены только для отдельных частных случаев. В общем случае применяют метод статистических испытаний (Монте-Карло).

Идея метода Монте-Карло: вместо того, чтобы описывать случайные явления с помощью аналитических зависимостей, производится \textbf{моделирование} случайного явления с помощью некоторой процедуры, которая дает <<случайный>> результат. Произведя такой розыгрыш очень большое количество раз, получаем статистический материал~--- множество реализаций случайного явления.

\image
{.5\textwidth}
{07/inc/carlo}
{}

Суть метода:

\begin{enumerate}
    \item Любым способом получаем два числа~--- $x_i$ и $y_i$, подчиняющихся равномерному распределению на интервале $[0, 1]$.
    \item Считаем, что одно число определяют координату точки по $Ox$, второе~--- по $Oy$.
    \item Анализируем, принадлежит ли точка заштрихованной поверхности. Если принадлежит, то в счетчик добавляем 1.
    \item Процедура генерации двух случайных чисел с заданным законом распределения и проверка принадлежности точки поверхности повторяется $n$ раз.
\end{enumerate}

Площадь можно определить как отношение количества точек, попавших в область, к общему числу сгенерированных точек.

Точность метода: ${\epsilon < \sqrt{\dfrac{1}{n}}}$.

Преимущества:

\begin{itemize}
    \item универсальность, обуславливающая возможность всестороннего статистического исследования объекта.
\end{itemize}

Недостатки:

\begin{itemize}
    \item для реализации нужны достаточно полные статистические сведения о параметрах;
    \item большой объем вычислений.
\end{itemize}

Уменьшение количества испытаний повышает скорость вычислений, но ухудшает их точность.

\subsection{Способы получения последовательности случайных чисел}

При имитационном моделировании сложных систем одним из основных вопросов является учет стохастических воздействий. Для этого способа моделирования нам необходимо большое число операций со случайными числами. Здесь характерна зависимость результатов от качества исходных (базовых) последовательностей.

На практике используется три основных способа генерации случайных чисел:

\begin{itemize}
    \item аппаратный;
    \item табличный (файловый);
    \item алгоритмический (программный).
\end{itemize}

\subsubsection{Аппаратный}

Случайные числа вырабатываются специальной электронной приставкой~--- генератором случайных чисел. Как правило, в качестве нее служит одно из внешних устройств. Реализация этого способа не требует дополнительных вычислительных операций (необходима только операция обращения к устройству). В качестве физического эффекта чаще всего используются электрические шумы в полупроводниковых приборах.

\image
{.75\textwidth}
{07/inc/noise}
{Структурная схема аппаратного генератора случайных чисел}

\vspace{.25cm}
\begin{itemize}
    \item[$+$] Запас чисел не ограничен.
    \item[$+$] Расходуется очень мало вычислительных операций.
    \item[$+$] Не занимает место в памяти.
    \item[$-$] Требуется периодическая проверка на случайность.
    \item[$-$] Нельзя воспроизводить последовательности.
    \item[$-$] Используются специальные устройства.
\end{itemize}

\subsubsection{Табличный}

Случайные числа оформляются в виде файла (таблицы).

\begin{itemize}
    \item[$+$] Требуется однократная проверка на случайность.
    \item[$+$] Можно воспроизводить последовательности.
    \item[$-$] Запас чисел ограничен.
    \item[$-$] Занимает место в памяти.
    \item[$-$] Требуется время на обращение.
\end{itemize}

\subsubsection{Алгоритмический}

Основан на специальных алгоритмах.

\begin{itemize}
    \item[$+$] Однократная проверка на случайность.
    \item[$+$] Можно многократно воспроизводить последовательности псевдослучайных чисел.
    \item[$+$] Занимает очень мало места в памяти.
    \item[$+$] Не требует специальных устройств.
    \item[$-$] Запас чисел ограничен периодом последовательности чисел.
    \item[$-$] Требуются существенные затраты вычислительных ресурсов.
\end{itemize}

Рассмотрим некоторую случайную величину $X$, которая может принимать любые значения из интервала $[a, b]$ и имеет плотность распределения $\frac{1}{b - a}$. Найти функцию распределения вероятности, математическое ожидание и дисперсию.
%
\begin{gather*}
    F(x) = \int\limits_{a}^b\dfrac{1}{b - a}dx = \begin{cases}
        0,                    & x \leqslant a     \\
        \dfrac{x - a}{b - a}, & a < x \leqslant b \\
        1,                    & x > b
    \end{cases}\\
    MX = \int\limits_{a}^{b}xf(x)dx = \dfrac{a + b}{2}\\
    DX = M(X^2) - (MX)^2 = \int\limits_{a}^{b}x^2f(x)dx = \dfrac{(b - a)^2}{12}
\end{gather*}
%
\subsection{Алгоритмический способ получения случайных чисел}

Для получения случайной величины из последовательности случайных величин с заданным законом распределения обычно используют одно или несколько значений, равномерно распределенных случайных чисел. Поэтому вопрос получения равномерного распределения имеет наиважнейшее значение.

Как правило, равномерное распределение получаем с помощью некоторого рекуррентного соотношения. Это означает, что каждое последующее число $a_{i + 1}$ образуется из предыдущего числа $a_i$ или из группы предыдущих чисел путем применения некоторого алгоритма, состоящего из арифметических и логических операций.

На языке FORTRAN необходимо было использовать следующие константы:

\begin{itemize}
    \item 1220703125;
    \item 2147483647;
    \item 2147483647 + 1 (почему не сложить сразу?);
    \item <...>.
\end{itemize}


\section{Лекция №6 (16.10.2023)}

Для получения соотношений, связывающих характеристики, описывающие функционирование Q-схем, вводят некоторые допущения относительно входных потоков, функций распределения, длительности обслуживания запросов, дисциплин обслуживания.

Для математического описания функционирования устройств, развивающихся в форме случайного процесса, может быть применен математический аппарат, разработанный в теории вероятностей для \textit{марковских} случайных процессов.

\begin{dd}
    Случайный процесс, протекающий в некоторой системе, называется \textit{марковским}, если для каждого момента времени вероятность любого состояния системы в будущем зависит только от состояния системы в настоящем и не зависит от того, когда и каким образом система пришла в это состояние\footnote{Примера таких систем в реальности нет.}.
\end{dd}

В марковском случайном процессе будущее развитие зависит только от настоящего состояния и не зависит от предыстории процесса. Для марковского случайного процесса составляют уравнения Колмогорова, представляющие собой соотношения следующего вида:
%
\begin{gather*}
    F(P'(t), P(t), \Lambda) = 0
\end{gather*}
%
Здесь $P(t)$~--- вероятность нахождения в состоянии для сложной системы, $\Lambda$~--- коэффициенты, показывающие, с какой скоростью система переходит из одного состояния в другое (интенсивность).

Для стационарного состояния:
%
\begin{gather*}
    \Phi(P(t), \Lambda) = 0\\
    Y = Y(P(t), \Lambda) = Y(P(\Lambda))
\end{gather*}
%
Последнее соотношение представляет собой зависимость выходных параметров от внутренних параметров модели и является \textit{базисной моделью}.
%
\begin{gather*}
    Y = Y(X, V, H)\\
    \Lambda= \Lambda(X, V, H)
\end{gather*}
%
Последняя формула~--- интерфейсная модель. Следовательно, математическая модель системы строится как совокупность базисной и интерфейсной моделей, что позволяет использовать одни и те же базисные модели для различных задач моделирования, осуществляя настройку на соответствующую задачу посредством изменения интерфейсной модели.

Пусть есть некоторая сложная система $S$.

\image
{.5\textwidth}
{06/inc/graph}
{}

Найдем вероятность $P_1(t)$, то есть вероятность того, что в момент времени $t$ система будет находиться в состоянии $S_1$. Придадим $t$ малое приращение ${\Delta t}$ и найдем вероятность того, что в момент ${t + \Delta t}$ система будет находиться в состоянии $S_1$.

Вероятность того, что система была в состоянии $S_1$ и за время $\Delta t$ не вышла из него, найдем как вероятность $P_1(t)$, то есть того, что в момент $t$ система была в состоянии $S_1$, на условную вероятность того, что, будучи в состоянии $S_1$ система не перейдет из него в $S_2$. Эта условная вероятность с точностью до бесконечно малых высших порядков равна
%
\begin{gather*}
    P_1(t + \Delta t) = P_1(t)\cdot(1-\lambda_{12}\Delta t) + P_3(t)\lambda_{32}\cdot\Delta t\\
    \dfrac{P_1(t + \Delta t) - P_1(t)}{\Delta t} = -P_1(t)\lambda_{12} + P_3(t)\lambda_{31}\\
    \lim_{\Delta t\rightarrow 0}\left(\dfrac{P_1(t + \Delta t) - P_1(t)}{\Delta t}\right) = -P_1(t)\lambda_{12} + P_3(t)\lambda_{31}\\
    P_1'(t) = -P_1(t)\lambda_{12} + P_3(t)\lambda_{31}
\end{gather*}
%
Состояние $S_2$ может быть определено как:

\begin{itemize}
    \item система перешла из $S_1$;
    \item система перешла из $S_4$;
    \item система, будучи в $S_2$, не перешла в $S_3$;
    \item система, будучи в $S_2$, не перешла в $S_4$.
\end{itemize}

В таком случае уравнения Колмогорова:
%
\begin{gather*}
    P_2'(t) = -P_2(t)\cdot(\lambda_{23} + \lambda_{24}) + P_1(t)\lambda_{12} + P_4(t)\lambda_{42}\\
    P_3'(t) = -P_3(t)\cdot(\lambda_{31} + \lambda_{34}) + P_2(t)\lambda_{23}\\
    P_4'(t) = -P_4(t)\lambda_{42} + P_2(t)\lambda_{24} + P_3(t)\lambda_{34}
\end{gather*}
%
Интегрирование данной системы дает искомые вероятности состояний как функции времени. Начальные условия берутся в зависимости от того, каково было начальное состояние системы. Например, если при ${t = 0}$ система находилась в состоянии $S_1$, то начальные условия:
%
\begin{gather*}
    P_1(0) = 1\\
    P_2(0) = 0\\
    P_3(0) = 0\\
    P_4(0) = 0
\end{gather*}
%
В эту же систему можно добавить уравнение (условие) нормировки: когда сумма всех вероятностей равна $1$ в любой момент времени.

Все уравнения Колмогорова построены по следующим правилам.

\begin{enumerate}
    \item В левой части каждого уравнения стоит производная вероятности, а правая часть содержит столько членов, сколько стрелок связано с данным состоянием.
    \item Если стрелка направлена из состояния, соответствующий член имеет знак <<$-$>>, если в состояние~--- <<$+$>>.
    \item Каждый член равен произведению плотности вероятности перехода (интенсивности), соответствующей данной стрелке, на вероятность того состояния, из которого выходит стрелка.
\end{enumerate}

Рассмотрим многоканальную систему массового обслуживания с отказами. Будем нумеровать состояния системы по числу занятых каналов (по числу заявок, связанных с системой):

\begin{itemize}
    \item $S_0$~--- все каналы свободны;
    \item $S_1$~--- занят 1 канал, остальные свободны;
    \item $S_k$~--- занято $k$ каналов, остальные свободны;
    \item $S_n$~--- заняты все $n$ каналов.
\end{itemize}

\image
{.75\textwidth}
{06/inc/flow}
{}

Разметим граф, то есть проставим у стрелок интенсивности соответствующих потоков событий. Поток заявок (слева направо) имеет интенсивность $\lambda$.

Определим интенсивность потока событий, которые переводят систему по стрелкам справа налево. Пусть система была в состоянии $S_1$, тогда, как только закончится обслуживание заявки, занимающей этот канал, система перейдет в состояние $S_0$. Данный поток обслуживания заявок имеет интенсивность $\mu$. Очевидно, что если заняты 2 канала, а не 1, то поток обслуживаний, переводящий систему из $S_2$ в $S_1$, будет вдвое интенсивней.
%
\begin{gather*}
    p_0 = \dfrac{1}{1 + \dfrac{\dfrac{\lambda}{\mu}}{1!} + \dfrac{\left(\dfrac{\lambda}{\mu}\right)^2}{2!} + \dots + \dfrac{\left(\dfrac{\lambda}{\mu}\right)^n}{n!}}\\
    p_k = \dfrac{\left(\dfrac{\lambda}{\mu}\right)^k}{k!}p_0\\
    \dfrac{\lambda}{\mu} = \rho\\
    p_{\text{отк}} = p_n = \dfrac{\rho^n}{n!}p_0\\
    q = 1 - p_n\\
    A = \lambda q = \lambda(1 - p_n)
\end{gather*}
%
Отказ: все $n$ каналов заняты ${p_{\text{отк}} = p_n}$. Относительная пропускная способность системы равна вероятности того, что заявка будет принята к обслуживанию.

Параметр $\mu$~--- это, как правило, функция технических характеристик объекта и средств реализаций (характеристики решаемых задач).

В простейшем случае, если время ввода-вывода каждого задания или процесса мало по сравнению со временем решения этого задания, принимают, что ${t = \frac{1}{\mu}}$, то есть среднее число операций, выполненных процессором, приходящихся на одно задание к среднему быстродействию процессора (речь идет о среднем времени).

\subsection{Лабораторная работа №2}

Предельные вероятности состояний $P_0\dots P_n$ характеризуют установившийся режим работы. В результате получаем $P_0$ и $P_k$.

\image
{.75\textwidth}
{06/inc/oscillation}
{}

Решив уравнения Колмогорова, получим $P$, в лабораторной работе нужно найти $t$.

Определить среднее относительное время пребывания системы в предельном стационарном состоянии. Интенсивности переходов из состояния в состояние задаются в виде матрицы максимальной размерности 10.

Определить вероятность нахождения системы в заданном состоянии.
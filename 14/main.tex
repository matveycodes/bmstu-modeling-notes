\section{Лекция №14 (18.12.2023)}

\subsection{Моделирование дискретных систем}

Сети Петри были разработаны и используются для моделирования сложных систем. С помощью различных модификаций этих сетей можно создать концептуальные модели и реализовать функционирование таких моделей <...> систем с независимыми элементами:

\begin{itemize}
    \item аппаратное и программное обеспечение;
    \item системы телекоммуникаций;
    \item физические, химические, социальные процессы.
\end{itemize}

При описании функционирования системы с помощью сетей Петри выделяют два понятия.

\begin{dd}
    Событие~--- это действие в системе. В сетях Петри моделируются \textit{переходами}.
\end{dd}

\begin{dd}
    Условие~--- это предикат или логическое описание системы, принимающее значение <<истина>> или <<ложь>>. Условия моделируются \textit{позициями}. Различают пред- и постусловия.
\end{dd}

\begin{dd}
    Предусловие~--- это условие до срабатывания перехода.
\end{dd}

\begin{dd}
    Постусловие~--- это условие после срабатывания перехода.
\end{dd}

\begin{dd}
    Переходы $t_i$ и $t_j$ находятся в \textit{конфликте}, если запуск одного из них блокирует запуск другого.
\end{dd}

\subsubsection{Простейшая система массового обслуживания}

Система имеет входной поток заданий. Пока она занята выполнением очередного задания, она не может ввести следующее задание.

Рассмотрим множество условий и событий, характеризующих данную систему:

\begin{itemize}
    \item условия:
          \begin{itemize}
              \item $p_1$~--- задание ждет обработки;
              \item $p_2$~--- задание обрабатывается;
              \item $p_3$~--- процессор свободен;
              \item $p_4$~--- задание ожидает вывода.
          \end{itemize}
    \item события:
          \begin{itemize}
              \item $t_1$~--- задание помещается во вторую очередь;
              \item $t_2$~--- начало выполнения задания;
              \item $t_3$~--- конец выполнения задания;
              \item $t_4$~--- задание выводится.
          \end{itemize}
\end{itemize}

\image
{\textwidth}
{14/inc/petri1}
{Сеть Петри (однопоточная СМО)}

Показанная на рисунке начальная маркировка соответствует состоянию, когда система свободна, а заявки обслужены (${M_0=[0,0,0,0]}$). При срабатывании перехода $t_1$ от внешнего источника поступает задание (${M_1=[0,1,0,1]}$). При этом может сработать переход $t_2$, что означает начало обслуживания задания и приводит к маркировке ${M_2=[0,1,0,0]}$. Затем срабатывает переход $t_3$, то есть окончание обслуживания задания и освобождение системы (${M_3=[0,0,1,1]}$).

Переходы $t_1$ и $t_4$ могут работать независимо.

\image
{.75\textwidth}
{14/inc/petri2}
{Сеть Петри (двухпоточная СМО)}

Конвейер состоит из следующих этапов:

\begin{itemize}
    \item сравнение порядков;
    \item выравнивание порядков;
    \item сложение мантисс;
    \item нормализация результата.
\end{itemize}

\image
{\textwidth}
{14/inc/petri3}
{Сеть Петри (конвейер)}

Условия:

\begin{itemize}
    \item $p_{i1}$~--- входной регистр свободен;
    \item $p_{i2}$~--- входной регистр заполнен;
    \item $p_{i3}$~--- блок занят;
    \item $p_{i4}$~--- выходной регистр свободен;
    \item $p_{i5}$~--- выходной регистр заполнен;
    \item $p_{i6}$~--- пересылка в следующий блок возможна.
\end{itemize}

События:

\begin{itemize}
    \item $t_{i1}$~--- начало работы $i$-го блока;
    \item $t_{i2}$~--- завершение работы $i$-го блока;
    \item $t_{i3}$~--- начало пересылки в ${(i+1)}$-ый блок;
    \item $t_{i4}$~--- завершение пересылки в ${(i+1)}$-ый блок.
\end{itemize}

% Язык GPSS (второй вопрос билета) — основные команды и т. д.
% Первый вопрос — теория сложных систем


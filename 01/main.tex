\section{Лекция №1 (11.09.2023)}

Литература:

\begin{enumerate}
    \item Советов, Яковлев <<Моделирование систем>> (2000+ г., лучше~--- последняя версия).
    \item Градов, Рудаков <<Компьютерное моделирование>>.
    \item Кельтон <<Имитационное моделирование>>.
\end{enumerate}

Моделирование удешевляет весь проект. Есть возможность реализовать <<фантастические>> условия.

\subsection{Философские аспекты моделирования}

Свойства объекта характеризуются данными.

Тип данных определяется множеством возможных значений и множеством операций над ним.

Методологическая основа моделирования~--- диалектический метод познания.

\begin{dd}
    Объект~--- это все то, на что направлена человеческая деятельность.
\end{dd}

Выработка методологии направлена на упорядочение получения и обработки информации об объектах, которые существуют вне нашего сознания и взаимодействуют между собой и внешней средой.

Научно-техническое развитие в любой области обычно идет следующим путем:

\begin{enumerate}
    \item Наблюдение и эксперимент.
    \item Теоретическое исследование.
    \item Организация производственного процесса.
\end{enumerate}

В научных исследованиях большую роль играют \textit{гипотезы}, то есть определенные предсказания, основывающиеся на небольшом количестве опытных данных, наблюдениях или догадках.

Быстрая и полная проверка гипотез может быть проведена в ходе специально поставленного эксперимента. В качестве метода суждения при формировании и проверке правильности гипотез большое значение имеет \textit{аналогия}.

\begin{dd}
    Аналогия~--- суждение о каком-либо частном сходстве двух объектов. Причем такое сходство может быть как существенным, так и не существенным. Существенность в основном зависит от уровня абстрагирования и в общем случае определяется конечной целью процесса исследования.
\end{dd}

Современная научная гипотеза создается как правило по аналогии с проверенными на практике положениями. Следовательно, именно аналогия связывает гипотезу и эксперимент.

Гипотезы и аналогии, отражающие реальный, объективно существующий мир, должны обладать наглядностью или сводиться к удобным для исследования логическим схемам. Такие логические схемы, упрощающие рассуждения и логические построения или позволяющие проводить эксперименты, уточняющие природу явлений, называются \textit{моделями}.

\begin{dd}
    Модель~--- это объект-заместитель объекта-оригинала, обеспечивающий изучение определенных свойств последнего.
\end{dd}

\begin{dd}
    Моделирование~--- это процесс замещения одного объекта другим с целью получения информации о важнейших свойствах объекта-оригинала с помощью объекта-модели.
\end{dd}

\subsection{Классификация видов моделирования}

\image
{\textwidth}
{01/inc/classification}
{Классификация видов моделирования сложных систем}

Самое главное при классификации~--- критерий классификации.

Детерминированное моделирование отражает процессы, в которых предполагается отсутствие случайных воздействий. Стохастическое отображает вероятностные процессы.

Статическое моделирование служит для описания поведения объектов в какой-либо момент времени, а динамическое отображает состояние системы во времени.

Дискретное~--- в дискретные моменты времени, непрерывное~--- на протяжении некоторого времени. Дискретно-непрерывное моделирование используется для случаев, когда необходимо выделить наличие как дискретных, так и непрерывных процессов.

Мысленное моделирование применяется для исследования объектов, которые либо практически нереализуемы в заданном интервале времени, либо существуют вне условий, возможных для их физического создания. При реальном моделировании исследуется реальный объект целиком или частично.

При наглядном моделировании на базе представлений человека о реальных объектах создаются различные наглядные модели, отображающие явления и процессы, протекающие в объекте.

Символическое моделирование представляет собой искусственный процесс создания логического объекта, который замещает реальный и выражает основные свойства его отношений с помощью определенной системы знаков или символов.

Под математическим моделированием будем понимать процесс установления соответствия данному реальному объекту некоторого математического объекта, называемого \textit{математической моделью}, и исследование этой модели, позволяющее получить характеристики объекта-оригинала. Вид математической модели зависит от реального объекта и целей моделирования.

Натурным моделированием называют проведение исследования на реальном объекте с последующей обработкой результатов эксперимента на основе теории подобия.

Физическое моделирование отличается от натурного тем, что исследование проводится на установках, которые сохраняют природу явлений и обладают физическим подобием. В процессе физического моделирования задаются некоторые характеристики внешней среды и исследуется поведение либо реального объекта, либо его модели при заданных или создаваемых искусственно воздействиях внешней среды.

В основу гипотетического моделирования исследователем закладывается некоторая гипотеза о закономерностях протекания процесса в реальном объекте, которая отражает уровень знаний исследователя об объекте и базируется на причинно-следственных связях между входом и выходом изучаемого объекта. Гипотетическое моделирование используется, когда знаний об объекте недостаточно для построения формальных моделей.

Аналоговое моделирование основывается на применении аналогий различных уровней. Наивысшим уровнем является полная аналогия, имеющая место только для достаточно простых объектов. С усложнением объекта используют аналогии последующих уровней, когда аналоговая модель отображает несколько либо только одну сторону функционирования объекта.

Мысленный макет может применяться в случаях, когда протекающие в реальном объекте процессы не поддаются физическому моделированию, либо может предшествовать проведению других видов моделирования. В основе построения мысленных макетов также лежат аналогии, однако обычно базирующиеся на причинно-следственных связях между явлениями и процессами в объекте.

Если ввести условное обозначение отдельных понятий (знаки), а также определенные операции между этими знаками, то можно реализовать знаковое моделирование и с помощью знаков отображать набор понятий~--- составлять отдельные цепочки из слов и предложений. Используя операции объединения, пересечения и дополнения теории множеств, можно в отдельных символах дать описание какого-то реального объекта.

В основе языкового моделирования лежит тезаурус~--- словарь, в котором каждому слову может соответствовать лишь единственное понятие.

Для аналитического моделирования характерно то, что процессы функционирования элементов системы записываются в виде некоторых функциональных соотношений или логических условий.

\begin{dd}
    Алгоритм~--- это упорядоченная последовательность действий, направленная на достижение конечной цели.
\end{dd}

Аналитическая модель может быть исследована тремя способами.

\begin{enumerate}
    \item Аналитическим. Здесь стремятся получить в общем виде зависимости от искомых характеристик.
    \item Численным. Когда нельзя решить уравнение в общем виде, то получают результаты для конкретных начальных данных.
    \item Качественным. Когда не имея решения можно найти свойства решения.
\end{enumerate}

При имитационном моделировании, реализующем модель алгоритма, воспроизводят процесс функционирования системы во времени, причем имитируются элементарные явления составляющих процесса с сохранением их логической структуры и последовательности протекания во времени, что позволяет по исходным данным получить сведения о состоянии процесса в определенные моменты времени, дающие возможность оценить характеристики системы.

Основным преимуществом имитационного моделирования по сравнению с аналитическим является возможность решения более сложных задач.

Имитационные модели позволяют достаточно просто учитывать такие факторы, как наличие дискретных и непрерывных элементов, нелинейные характеристики элементов, многочисленные случайные воздействия~--- что трудно описать аналитически.

Комбинированное (аналитико-имитационное) моделирование позволяет объединить достоинства аналитического и имитационного моделирования. При построении комбинированных моделей проводится предварительная декомпозиция процесса функционирования объекта на составляющие подпроцессы, и для тех из них, где это возможно, используются аналитические модели, а для остальных подпроцессов строятся имитационные модели.
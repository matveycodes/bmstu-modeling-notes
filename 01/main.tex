\section{Лекция №1 (11.09.2023)}

Литература:

\begin{enumerate}
    \item Советов, Яковлев <<Моделирование систем>> (2000+ г., лучше~--- последняя версия).
    \item Градов, Рудаков <<Компьютерное моделирование>>.
    \item Кельтон <<Имитационное моделирование>>.
\end{enumerate}

GPSS команда \texttt{ANOVA}. GPSS команда \texttt{PLOT}.

Моделирование удешевляет весь проект. Есть возможность реализовать <<фантастические>> условия.

\subsection{Философские аспекты моделирования}

Свойства объекта характеризуются данными.

Тип данных определяется множеством возможных значений и множеством операций над ним.

Методологическая основа моделирования~--- диалектический метод познания.

\begin{dd}
    Объект~--- это всё то, на что направлена человеческая деятельность.
\end{dd}

Выработка методологии направлена на упорядочение получения и обработки информации об объектах, которые существуют вне нашего сознания и взаимодействуют между собой и внешней средой.

Научно-техническое развитие в любой области обычно идёт следующим путём: наблюдение и эксперимент, теоретическое исследование и организация производственного процесса.

В научных исследованиях большую роль играют \textit{гипотезы}, то есть определённые предсказания, основывающиеся на небольшом количестве опытных данных, наблюдениях или догадках.

Быстрая и полная проверка гипотез может быть проведена в ходе специально поставленного эксперимента. В качестве метода суждения при формировании и проверке правильности гипотез большое значение имеет \textit{аналогия}.

\begin{dd}
    Аналогия~--- суждение о каком-либо частном сходстве двух объектов. Причём такое сходство может быть как существенным, так и не существенным. Существенность в основном зависит от уровня абстрагирования и в общем случае определяется конечной целью процесса исследования.
\end{dd}

Современная научная гипотеза создаётся как правило по аналогии с проверенными на практике положениями. Следовательно, именно аналогия связывает гипотезу и эксперимент.

Гипотезы и аналогии, отражающие реальный, объективно существующий мир, должны обладать наглядностью или сводиться к удобным для исследования логическим схемам. Такие логические схемы, упрощающие рассуждения и логические построения или позволяющие проводить эксперименты, уточняющие природу явлений, называются \textit{моделями}.

\begin{dd}
    Модель~--- это объект, заместитель объекта-оригинала, обеспечивающий изучение некоторых свойств оригинала.
\end{dd}

\begin{dd}
    Моделирование~--- это процесс замещения одного объекта другим с целью получения информации о важнейших свойствах объекта-оригинала с помощью объекта-модели.
\end{dd}

\image
{\textwidth}
{01/inc/classification}
{Классификация видов моделирования сложных систем}

Самое главное при классификации~--- критерий классификации.

Детерминированное моделирование отражает такие процессы, в которых предполагается отсутствие всяких случайных воздействий. Стохастическое отображает вероятностные процессы.

Статическое служит для описания поведения объектов в какой-либо момент времени, а динамическое отображает состояние системы во времени.

Дискретное~--- в дискретные моменты времени, непрерывные~--- на протяжении некоторого времени.

\begin{dd}
    Под математическим моделированием будем понимать процесс установления данному реальному объекту некоторого математического объекта, называемого математеческой моделью, и исследование этой модели, позволяющее получить характеристики объекта-оригинала.
\end{dd}

Вид математической модели зависит от реального объекта и целей моделирования.

Для аналитического моделирования характерно то, что процессы функционирования элементов системы записываются в виде некоторых функциональных соотношений или логических условий.

\begin{dd}
    Алгоритм~--- это упорядоченная последовательность действий, направленная на достижение конечной цели.
\end{dd}

Аналитическая модель можеть быть исследована тремя способами.

\begin{enumerate}
    \item Аналитическим. Здесь стремятся получить в общем виде зависимости от искомых характеристик.
    \item Численным. Когда нельзя решить уравнение в общем виде, то получают результаты для конкретных начальных данных.
    \item Качественным. Когда не имея решения можно найти свойства решения.
\end{enumerate}

При имитационном моделировании, реализующем модель алгоритма, воспроизводят процесс функционирования системы во времени, причём имитируются элементарные явления составляющих процесса с сохранением их логической структуры и последовательности протекания во времени, что позволяет по исходным данным получить сведения о состоянии процесса в определённые моменты времени, дающие возможность оценить характеристики системы.

Основным преимуществом имитационного моделирования по сравнению с аналитическим является возможность решения более сложных задач.

Имитационные модели позволяют достаточно просто учитывать такие факторы, как наличие дискретных и непрерывных элементов, нелинейные характеристики элементов, многочисленные случайные воздействия~--- что трудно описать аналитически.
\section{Лекция №9 (13.11.2023)}

\subsection{Методика построения программной модели}

Для разработки программной модели исходная система должна быть представлена как стохастическая система массового обслуживания. Это объясняется тем, что информация от внешней среды поступает в случайные моменты времени, длительность обработки различных типов информации может быть в общем случае различна. Следовательно, внешняя среда является как бы генератором сообщений, а комплекс вычислительных средств~--- обслуживающими устройствами.

\image
{\textwidth}
{09/inc/structure}
{Структурная схема программно-алгоритмического комплекса}

Источники информации выдают на вход буферной памяти независимо друг от друга сообщения. Закон появления сообщений произволен, но, как правило, формализован и задан.

В памяти сообщения записываются подряд и выбираются по одному на обслуживающий аппарат по принципу FIFO. Длительность обработки одного сообщения также может быть случайна, но закон обработки должен быть задан.

Быстродействие обслуживающего аппарата ограничено, поэтому возникает очередь на обработку.

Смысл программы синхронизации (управления) заключается в \textit{протяжке} модельного времени.

\subsubsection{Моделирование потока сообщений}

Поток сообщений обычно моделируется моментами появления очередного сообщения в потоке, на которое мы продвигаем модельное время.

\begin{table}[H]
    \renewcommand{\arraystretch}{1.5}
    \caption{Формулы генерации моментов времени}
    \begin{tabularx}{\textwidth} {
            >{\raggedright\arraybackslash}X
            >{\centering\arraybackslash}X}
        \toprule
        \textbf{Распределение} & \textbf{Момент времени $t_i$}                                                                \\
        \midrule
        Равномерное            & ${a + (b - a)R}$                                                                             \\
        Экспоненциальное       & ${\dfrac{1}{\lambda}\ln(1 - R)}$                                                             \\
        Нормальное             & ${\sigma_t\sqrt{\dfrac{12}{n}}}\left(\sum\limits_{i = 1}^n R_i - \dfrac{n}{2} \right) + M_x$ \\
        Эрланга                & ${-\dfrac{1}{k\lambda}}\sum\limits_{i=1}^k\ln(1-R_i)$                                        \\
        \bottomrule
    \end{tabularx}
\end{table}

\subsubsection{Моделирование работы обслуживающего аппарата}

Программа, моделирующая работу обслуживающего аппарата, --- это набор функций, вырабатывающих случайные отрезки времени, соответствующие длительностям обслуживания.
%
\begin{gather*}
    T_{\text{обр}} = M_x + \left(\sum\limits_{i = 1}^{12}R_i - 6\right)\delta_x,\,\text{где } X\sim N(M_x, \delta_x)
\end{gather*}
%

\subsubsection{Моделирование работы абонентов}

Абонент может рассматриваться как обслуживающий аппарат, поток информации на который поступает от процессора.

Для имитации работы необходимо написать программу выработки длительности обслуживания. Кроме того, абонент сам может быть источником заявок на те или иные ресурсы вычислительной системы. Эти заявки могут имитироваться с помощью генератора сообщений по заданному закону.

\subsubsection{Моделирование работы буферной памяти}

Моделирование начинаем с разработки концептуальной модели.

\begin{dd}
    Память --- электромеханическое устройство, предназначеное для хранения, записи и чтения информации.
\end{dd}

Блок буферной памяти должен производить запись и считывание числа, выдавать сигнал переполнения и отсутствия данных, в любой момент модельного времени располагать сведения о количестве находящихся в блоке требований.

Сама запоминающая среда в простейшем случае имитируется одномерным массивом, размер которого определяется объемом. Каждый элемент этого массива может быть либо свободен, либо занят, и в этом случае в качестве эквивалента требований ему присваивается значение времени появления этого требования.

\image
{\textwidth}
{09/inc/memory}
{Структурная схема памяти}

\image
{\textwidth}
{09/inc/memory-algorithm}
{Схема алгоритма буферной памяти}

Здесь:

\begin{itemize}
    \item P --- массив обращения;
    \item WYB --- признак обращения к буферной памяти (0 --- режим записи, 1 --- режим
          выбора сообщения);
    \item NP --- число сообщений в памяти;
    \item LM --- объём буферной памяти;
    \item NPOS --- номер последнего сообщения в памяти;
    \item NPER --- номер первого сообщения в памяти;
    \item POLN --- признак переполнения (1 --- нет свободных ячеек);
    \item X --- ячейка для сообщения;
    \item PUST --- признак отсутствия сообщения (1 --- в памяти нет сообщения).
\end{itemize}

\subsubsection{Моделирование программы сбора статистики}

Задача блока статистики заключается в накоплении численных значений, необходимых для вычисления статистических оценок заданных параметров работы моделируемой вычислительной системы.

При моделировании простейшей СМО интерес представляет среднее время ожидания в очереди. Для каждого сообщения время ожидания в очереди. Для каждого сообщения время ожидания в очереди равно разности между моментом времени, когда оно было выбрано на обработку, и моментом, когда оно пришло в систему от источника информации.

Коэффицент загрузки обслуживающего аппарата определяется как отношение времени работы обслуживающего аппарата к общему времени моделирования.

Подсчитав число потерянных сообщений, можно определить вероятность отказа или потери сообщения:
%
\begin{gather*}
    \dfrac{N_{\text{потер}}}{N_\text{потер} + N_\text{обраб}}
\end{gather*}
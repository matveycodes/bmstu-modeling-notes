\section{Лекция №11 (27.11.2023)}

\subsection{Моделирование систем и языки моделирования}

Алгоритмические языки при моделировании систем служат вспомогательным аппаратом разработки компьютерной реализации и анализа характеристик систем.

Отсюда огромнейшее значение имеет вопрос правильного выбора языка.

Язык моделирования, как и любой другой язык, имеет тщательно разработанную систему абстракций. Причем эти абстракции должны быть полностью направлены на анализ и контроль правильности функционирования объекта моделирования.

Качество языков моделирования характеризуется:

\begin{itemize}
    \item удобством описания процесса функционирования;
    \item удобством ввода исходных данных, варьирования структуры, алгоритмов, задания параметров модели;
    \item анализом вывода результатов моделирования;
    \item простотой отладки и контроля работы моделирующей программы;
    \item доступностью восприятия и использования языка.
\end{itemize}

Все современные языки моделирования определяют поведение систем во времени с помощью событийного алгоритма.

\image
{.9\textwidth}
{11/inc/classification}
{Классификация языков моделирования}

Непрерывное представление систем сводится к представлению дифференциальных уравнений, с помощью которых устанавливается связь вход-выход. Если выходные переменные модели принимают дискретные значения, то уравнения являются разностными.

\subsection{Формальное описание динамики моделируемого объекта}

Будем считать, что любая работа в системе совершается путем выполнения \textit{активностей}. Принимают, что активность является наименьшей единицей работы. Её рассматривают как единый дискретный шаг.

\begin{dd}
    Активность~--- это единый динамический объект, указывающий на совершение единицы работы.
\end{dd}

\begin{dd}
    Процесс~--- это логически связанный набор активностей.
\end{dd}

Пример процесса~--- запись в память.

\begin{dd}
    Событие~--- это мгновенное изменение состояния объекта.
\end{dd}

Рассмотренные объекты (активности, процессы, события) являются конструктивными элементами для динамического описания поведения системы. На их основе строятся языки моделирования. В то время, когда динамическое поведение системы формируется в результате выполнения большого числа взаимодействующих процессов, сами эти процессы образуют небольшое число классов. Следовательно, чтобы описать поведение систем, достаточно описать поведение каждого класса процессов и задать значения атрибутов для конкретных процессов.

Построение модели состоит из двух основных задач.

\begin{enumerate}
    \item Описание правил, задающих виды процессов, происходящих в системе.
    \item Описание значений атрибутов таких процессов или правил генерации этих значений. При этом система описывается на определенном уровне детализации в терминах множества описаний процессов, каждый из которых включает множество правил и условий возбуждения активностей.
\end{enumerate}

Модель служит для отображения временн\'{о}го поведения системы, поэтому даже языки дискретного моделирования должны обладать средствами отображения времени.

\subsection{SIMSCRIPT}

Моделирующая программа организована в виде секций событий. Событие состоит из набора операций, которые выполняются после завершения активности.

\subsection{SIMULA}

Моделирующая программа организуется в виде набора описаний процесса, каждый из которых описывает один класс. Описание процесса устанавливает атрибуты активности. Протяжка модельного времени осуществляется с помощью списка будущих событий.

\subsection{GPSS}

Можно отнести к группе языков, ориентированных на процессы. Его используют для описания пространственного движения объектов. Такие динамические объекты в языке GPSS называются \textit{транзактами} и представляют собой элементы потока.

\begin{table}[H]
    \renewcommand{\arraystretch}{1.5}
    \caption{Сравнение универсальных и специализированных языков моделирования}
    \begin{tabularx}{\textwidth} {
            >{\raggedright\arraybackslash}X
            >{\raggedright\arraybackslash}X}
        \toprule
        \multicolumn{1}{c}{\textbf{Преимущества}}                                                                        & \multicolumn{1}{c}{\textbf{Недостатки}}                          \\
        \midrule
        \multicolumn{2}{c}{\textbf{Универсальные}}                                                                                                                                          \\\midrule
        Минимум ограничений на выходной формат                                                                           & Значительное время, затрачиваемое на программирование            \\
        Широкое распространение                                                                                          & Значительное время, затрачиваемое на отладку                     \\\midrule
        \multicolumn{2}{c}{\textbf{Специализированные}}                                                                                                                                     \\\midrule
        Меньше затраты времени на программирование                                                                       & Необходимость точно придерживаться ограничений на форматы данных \\
        Более эффективные методы выявления ошибок                                                                        & Меньшая гибкость модели                                          \\
        Краткость, точность понятий, характеризующих имитируемые конструкции                                             &                                                                  \\
        Возможность заранее строить стандартные блоки, которые могут использоваться в любой имитационной модели          &                                                                  \\
        Автоматическое формирование определенных типов данных, необходимых именно в процессе имитационного моделирования &                                                                  \\
        Удобство накопления и представления выходной информации                                                          &                                                                  \\
        Эффективное использование ресурсов                                                                               &                                                                  \\
        \bottomrule
    \end{tabularx}
\end{table}

Возможности программного обеспечивания и имитационного моделирования:

\begin{itemize}
    \item анимация и динамическая графика:
          \begin{itemize}
              \item представление сути имитационного моделирования;
              \item отладка моделируемой программы;
              \item обучение обслуживающего персонала.
          \end{itemize}
    \item статистические возможности:
          \begin{itemize}
              \item генераторы псевдослучайных чисел.
          \end{itemize}
\end{itemize}

Существует два типа анимаций:

\begin{itemize}
    \item совместная (осуществляется во время прогонки модели);
    \item реальная.
\end{itemize}

\subsection{Объектно-ориентированное моделирование}

Преимущества:

\begin{itemize}
    \item возможность повторного использования объектов и удобный способ их изменения;
    \item реализация иерархического подхода;
    \item упрощение изменения модели;
    \item возможность декомпозиции задачи на подзадачи.
\end{itemize}

Недостаток~--- трудность изучения.

\subsection{Лабораторная работа №5}

Моделируем информационный центр. В информационный центр приходят клиенты (пользователи) через интервал времени ${10\pm 2}$ минуты. Если все три имеющихся оператора заняты, клиенту отказывают в обслуживании.

Операторы имеют разную производительность и могут обеспечивать обслуживание среднего запроса от пользователя за ${20\pm 5}$, ${40\pm 10}$ и ${40\pm 20}$ ед. времени (минут). Клиенты стараются занять свободного оператора с максимальной производительностью.

Полученные запросы сдаются в накопитель, откуда выбираются на обработку. На первый компьютер~--- от первого и второго операторов, на второй~--- от третьего. Время обработки запроса в компьютерах~--- $15$ и $30$ минут соответственно.

Смоделировать процесс \textbf{обработки} $300$ запросов. Определить вероятность отказа.

\image
{\textwidth}
{11/inc/lab}
{Концептуальная модель}

В процессе взаимодействия клиентов с информационным центром возможны:

\begin{itemize}
    \item режим нормального обслуживания, когда клиент выбирает одного из свободных операторов, отдавая предпочтение тому, у которого меньше номер.
    \item режим отказа в обслуживании, когда все три оператора заняты.
\end{itemize}

\image
{.75\textwidth}
{11/inc/lab-detailed}
{}

Переменные и уравнения имитационной модели: эндогенные (время обработки задания $i$-ым оператором и время решения задания $j$-ым компьютером) и экзогенные (число обслуженных клиентов $N_0$ и число клиентов, получивших отказ,~--- $N_1$).

Уравнение модели имеет вид (\ref{eq:lab5}).
%
\begin{gather}
    \label{eq:lab5}
    P_{\text{отк}} = \dfrac{N_1}{N_0 + N_1}
\end{gather}
%
За единицу системного времени выбрать $\frac{1}{100}$ минуты.
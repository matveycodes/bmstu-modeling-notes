\section{Лекция №3 (25.09.2023)}

\subsection{Гибридные вычислительные машины}

Гибридная вычислительная машина объединяет в себе узлы и блоки типовых или специализированных аналоговых и цифровых вычислительных машин с использованием \textbf{различных форм представления информации} и методов ее переработки.

\image
{.5\textwidth}
{03/inc/scheme}
{}

Для гибридной вычислительной техники характерным является:

\begin{itemize}
    \item различные преобразователи форм представления информации;
    \item прецизионные коммутаторы непрерывных сигналов;
    \item запоминающие устройства непрерывных сигналов;
    \item устройства сравнения.
\end{itemize}

Создание гибридной вычислительной машины ставит своей целью сочетать быстродействие аналоговой и точность цифровой техники. Например, путем многократного использования операционных блоков посредством использования цифрового управления.

Применение гибридных вычислительных машин:

\begin{itemize}
    \item моделирование дискретных систем и случайных процессов;
    \item решение задач оптимизации;
    \item исследования в области управления подвижными объектами;
    \item моделирование систем <<человек и машина>>.
\end{itemize}

Можно выделить три основных направления развития гибридных вычислительных машин.

\begin{enumerate}
    \item На основе дискретно управляемых регистров.
    \item Разрядно-аналоговые. Высокая точность за счет цифровой формы представления сигнала. Перерабатываем информацию аналоговым способом.
    \item Цифровые интегрирующие компьютеры. Основная операция~--- не суммирование, а интегрирование.
\end{enumerate}

\begin{table}[H]
    \renewcommand{\arraystretch}{1.5}
    \caption{Сравнение аналоговой и цифровой техники}
    \begin{tabularx}{\textwidth} {
            >{\raggedright\arraybackslash}X
            >{\centering\arraybackslash}X
            >{\centering\arraybackslash}X}
        \toprule
                                                        & \textbf{АВМ}                                    & \textbf{ЭВМ}                         \\
        \midrule
        \textbf{Тип информации}                         & Непрерывный                                     & Дискретный                           \\
        \textbf{Изменение значений}                     & Величиной напряжения                            & Числовым значением                   \\
        \textbf{Базовые операции}                       & Арифметические операции и интегрирование        & Арифметические операции              \\
        \textbf{Принцип вычисления}                     & Высокопараллельный                              & Ограничен                            \\
        \textbf{Динамическое изменение решаемой задачи} & Посредством системы коммутации                  & В диалоговом режиме                  \\
        \textbf{Требования к пользователю}              & Профессиональные знания, методика моделирования & Знание основ ПО ЭВМ                  \\
        \textbf{Уровень формализации задачи}            & Ограничен моделью решаемой задачи               & Высокий                              \\
        \textbf{Способность к решению задач}            & Ограничена                                      & Высокая                              \\
        \textbf{Точность вычисления}                    & Ограничена ($10^{-4}$)                          & Ограничена разрядностью ($10^{-40}$) \\
        \textbf{Диапазон представления чисел}           & $1\dots10^4$                                    & Зависит от разрядности               \\
        \textbf{Класс решаемых задач}                   & Алгебраические и дифф. уравнения                & Любые                                \\
        \textbf{Специальные функции}                    & Ограниченный набор                              & Неограниченный набор                 \\
        \textbf{Уровень миниатюризации}                 & Ограничен                                       & Высокий                              \\
        \textbf{Сфера применения}                       & Ограничена                                      & Практически любая                    \\
        \bottomrule
    \end{tabularx}
\end{table}

\subsection{Основы теории моделирования}

\image
{.5\textwidth}
{03/inc/system}
{}

Модель объектного моделирования можно представить в виде множества величин, описывающих процесс функционирования реальной системы и образующих в общем случае следующие подмножества.

\begin{enumerate}
    \item Совокупность входных воздействий $x_i\in X$, $i = \overline{1, n_X}$.
    \item Совокупность воздействий внешней среды $v_l\in V$, $l = \overline{1, n_V}$.
    \item Совокупность внутренних (собственных) параметров системы $h_k\in H$, $k = \overline{1, n_H}$.
    \item Совокупность выходных характеристик системы $y_j\in Y$, $j = \overline{1, n_Y}$.
\end{enumerate}

В общем случае $x_i$, $v_l$, $h_k$ и $y_j$ являются элементами непересекающихся подмножеств и содержат как детерминированные, так и стохастические составляющие. При моделировании функционирования такой системы входные воздействия, воздействия внешней среды и внутренние параметры являются независимыми (экзогенными):
%
\begin{gather*}
    \vv{x(t)} = (x_1(t), x_2(t), \dots, x_{n_X}(t))\\
    \vv{v(t)} = (v_1(t), v_2(t), \dots, v_{n_V}(t))\\
    \vv{h(t)} = (h_1(t), h_2(t), \dots, h_{n_H}(t))
\end{gather*}
%
Выходные характеристики системы являются зависимыми (эндогенными):
%
\begin{gather*}
    \vv{\overline{y}(t)} = (y_1(t), y_2(t), \dots)
\end{gather*}
%
Процесс функционирования системы $S$ описывается во времени некоторым оператором, который в общем случае преобразует независимые переменные в зависимые:
%
\begin{gather*}
    \vv{y(t)} = F_S(\vv{x}, \vv{v}, \vv{h}, t)
\end{gather*}
%
Эта зависимость называется \textit{законом функционирования системы}. В общем случае она может быть задана в виде функции, функционала, логических условий, в алгоритмическом или табличном виде.

\begin{dd}
    Алгоритм функционирования~--- метод получения выходных характеристик с учетом входных воздействий, воздействий внешней среды и соответствующих внутренних параметров системы.
\end{dd}

Закон функционирования может быть получен и через состояния системы, то есть через свойства системы в конкретные моменты времени:
%
\begin{gather*}
    \vv{z} = (z_1(t), z_2(t), \dots, z_p(t))
\end{gather*}
%
Если рассматривать процесс функционирования системы как последовательную смену состояний $z_1, z_2, \dots, z_p$, то они могут быть интерпретированы как координаты точки в $p$-мерном пространстве, причем каждой реализации процесса будет соответствовать некоторая фазовая траектория.

\begin{dd}
    Пространство состояний объекта моделирования~--- это совокупность всех значений состояний.
\end{dd}

Состояние системы в некоторый момент времени $t$ полностью определяется некоторыми начальными условиями, входными воздействиями, внутренними параметрами, воздействиями внешней среды, которые имели место за время $t - t_0$:
%
\begin{gather*}
    \vv{z^0} = (z_1^0, z_2^0, \dots, z_p^0)\\
    \vv{z(t)} = \Phi(\vv{z^0}, \vv{x}, \vv{v}, \vv{h}, t)\\
    \vv{y(t)} = F(\vv{z}, t)
\end{gather*}
%
Закон функционирования через состояние:
%
\begin{gather*}
    \vv{y(t)} = F(\Phi(\vv{z^0}, \vv{x}, \vv{h}, t))
\end{gather*}
%
В общем случае время в модели может быть непрерывным на некотором интервале, а может быть дискретным (квантованным на некотором отрезке).

\begin{dd}
    Под математической моделью реальной системы понимают конечное множество переменных $x(t)$, $v(t)$ и $h(t)$ вместе с математическими связями между ними.
\end{dd}

\subsection{Типовые математические схемы}

В практике моделирования на первоначальных этапах формализации объекта используют \textit{типовые математические схемы}, к которым можно отнести хорошо разработанные и многократно проверенные на практике математические объекты.

\begin{table}
    \renewcommand{\arraystretch}{1.5}
    \begin{tabularx}{\textwidth} {
            >{\raggedright\arraybackslash}X
            >{\centering\arraybackslash}X
            >{\centering\arraybackslash}X}
        \toprule
        \textbf{Процесс функционирования системы} & \textbf{Типовая математическая схема} & \textbf{Обозначение} \\
        \midrule
        Непрерывно-детерминированный подход       & стандартные ДУ                        & D-схема              \\
        Дискретно-детерминированный подход        & конечные автоматы                     & F-схема              \\
        Дискретно-стохастический подход           & вероятностные автоматы                & P-схема              \\
        Непрерывно-стохастический подход          & система массового обслуживания        & Q-схема              \\
        Обобщенные (универсальный)                & агрегативная система                  & A-схема              \\
        \bottomrule
    \end{tabularx}
\end{table}

<...>

Таким образом, использование D-схем позволяет формализовать процесс функционирования непрерывно-детерминированных систем и оценить их основные характеристики, применяя аналитический или имитационный подход, реализованный в виде соответствующего языка моделирования непрерывных систем или АВМ/ГВМ.
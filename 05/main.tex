\section{Лекция №5 (09.10.2023)}

\subsection{Лабораторная работа №1}

Изучить равномерное распределение от $a$ до $b$ и распределение по варианту. Выслать отчет (титульный лист, графики функций, скриншот запущенной программы) до четверга 14:00.

\subsection{Уровни проектирования}

\begin{enumerate}
    \item Системный уровень. Рассматриваем всю вычислительную систему как единую модель, в качестве элементов~--- процессор, память, все внешние устройства. Используемые модели: вероятностные автоматы, системы массового обслуживания, графовые структуры (сети Петр\'{и}), агрегативные модели. Решаются вопросы, связанные с производительностью системы.
    \item Функционально-логический уровень (ФЛУП):
          \begin{enumerate}
              \item Уровень регистровых передач (РП). Рассматриваем, например, из чего состоит процессор. Типовая математическая схема~--- автоматы или дискретные сети. На этом уровне смотрим, как реализуется команда процессора.
              \item Логический уровень (ЛУ). Рассматриваем вопросы логики преобразования информации, разбиваем машинную команду на алгоритмы реализации этой операции в двоичной системе счисления.
          \end{enumerate}
          Рассматриваем отдельные части системы.
    \item Схемотехнический. Дискретные элементы рассматриваются в виде электронных реализаций (электронных схем). Математические модели~--- интегро-дифференциальные уравнения. Рассматриваются вопросы различных задержек и искажений сигнала.
    \item Конструкторский уровень. Речь идет о <<железе>>, основной вопрос~--- решение размещения всех элементов на плате. Самые сложные модели связаны с вопросами охлаждения и отведения тепла.
\end{enumerate}

\subsection{Моделирование на системном уровне}

Используются модели вычислительных систем и сетей.

\begin{dd}
    Вычислительная система~--- комплекс аппаратных и программных средств, которые в совокупности выполняют определенные рабочие функции.
\end{dd}

\begin{dd}
    Операционная система~--- набор ручных и автоматических процедур, которые позволяют группе людей эффективно использовать вычислительную установку.
\end{dd}

Основная черта операционных систем~--- согласованность работы управляющих программ.

\begin{dd}
    Коллектив пользователей~--- сообщество людей, которое использует систему для удовлетворения своих нужд по обработке информации.
\end{dd}

\begin{dd}
    Входная информация~--- программы, данные, которые создаются коллективом пользователей, и называются \textit{рабочей нагрузкой}.
\end{dd}

\image
{\textwidth}
{05/inc/compsystem}
{}

\begin{dd}
    Индекс производительности~--- описатель, который используется для представления производительности в самом широком смысле слова.
\end{dd}

Различают количественные и качественные индексы производительности. Качественными, например, являются удобство использования, субъективная легкость использования системы, мощность системы команд. Основными количественными индексами являются:

\begin{itemize}
    \item пропускная способность~--- объем информации, обрабатываемый в единицу времени;
    \item время ответа (реакции)~--- время между предъявлением системе входных данных и появлением соответствующей выходной информации.
\end{itemize}

\begin{dd}
    Коэффициент использования оборудования~--- отношение времени использования заданной части оборудования к определенному временному интервалу.
\end{dd}

Концептуальная модель вычислительной системы включает также сведения о выходных и конструктивных параметрах системы, ее структуре, особенностях работы каждого элемента, характере взаимодействия между ресурсами.

Основные решаемые задачи:

\begin{itemize}
    \item определение принципов организации вычислительной системы;
    \item выбор архитектуры;
    \item уточнение функций вычислительной системы;
    \item разделение функций на подфункции, реализуемые аппаратным и программным путем;
    \item разработка структурной схемы (определение состава устройств и способов их взаимодействия);
    \item определение требований к выходным параметрам устройств.
\end{itemize}

Особенности непрерывного стохастического подхода рассмотрим на примере использования в качестве типовой математической схемы систем массового обслуживания (queueing-систем). При этом исследуемая система формализуется как некоторая система обслуживания (например, компьютер и поток заявок). Характерным для таких объектов является \textbf{случайное} появление заявок на обслуживание и \textbf{случайное} завершение обслуживания. Случайность относится только к моментам времени.

В любом элементарном акте обслуживания можно выделить две основные составляющие: ожидание обслуживания и само обслуживание.

\image
{\textwidth}
{05/inc/machine}
{$i$-ый прибор обслуживания}

В накопителе $H_i$ может находиться от $0$ до емкости накопителя заявок.

\begin{dd}
    Поток событий~--- последовательность событий, происходящих одно за другим в случайные моменты времени.
\end{dd}

\begin{dd}
    Поток называется \textit{однородным}, если он характеризуется только моментами поступления (вызывающими моментами).
\end{dd}

\begin{dd}
    Поток называется \textit{неоднородным}, если он задается последовательностью, в которой кроме вызывающих моментов есть набор признаков события (например, приоритетность).
\end{dd}

\begin{dd}
    Если интервалы времени между сообщениями не зависят между собой и являются случайными, то поток имеет \textit{ограниченное последействие}.
\end{dd}

\begin{dd}
    Поток событий называется \textit{ординарным}, если вероятность того, что на малый интервал времени $\Delta t$ попадает больше одного события, пренебрежительно мала по сравнению с вероятностью того, что на этот интервал времени попадает ровно одно событие.
\end{dd}

\begin{dd}
    Поток называется \textit{стационарным}, если вероятность появления того или иного числа событий на интервале времени зависит лишь от длины участка и не зависит от того, где на оси времени располагается этот участок.
\end{dd}

Среднее число сообщений, поступающих в единицу времени,~--- интенсивность ординарного потока:
%
\begin{gather*}
    \lambda(t) = \lim\limits_{\Delta t\rightarrow 0}\dfrac{P_1(t, \Delta t)}{\Delta t}
\end{gather*}
%
Для стационарного потока его интенсивность не зависит от времени и представляет собой постоянное значение, равное числу средних событий.

В $i$-ом обслуживающем приборе имеем:

\begin{itemize}
    \item поток заявок $w_i$~--- интервалы времени между появлением заявок на входе команд;
    \item поток обслуживания $u_i$~--- интервалы времени между началом и окончанием обслуживания заявки.
\end{itemize}

Заявки, обслуженные каналом, и заявки, покинувшие прибор необслуженными, образуют выходной поток $y_i$.

Процесс функционирования $i$-ого прибора обслуживания можно представить как процесс изменения состояний его элементов во времени. Переход в новое состояние для $i$-ого прибора означает изменение количества заявок, которые могут находиться либо в канале, либо в накопителе. Вектор состояния этого прибора состоит из двух компонентов: состояния накопителя и состояния канала.

Состояние канала~--- канал свободен или занят обслуживанием заявки.

В практике моделирования элементарные Q-схемы обычно объединяют, при этом, если каналы различных приборов обслуживания соединены параллельно, то имеет место \textit{многоканальное} обслуживание, а если последовательно~--- \textit{многофазное}. Следовательно, для задания Q-схемы необходимо использовать оператор сопряжения, отражающий взаимосвязь элементов структуры.

Различают разомкнутые (выходной поток не может поступать заново на обслуживание) и замкнутые Q-схемы.

Собственные внутренние параметры Q-схемы:

\begin{itemize}
    \item количество фаз;
    \item количество каналов в каждой фазе;
    \item количество накопителей в каждой фазе;
    \item емкости накопителей.
\end{itemize}

Если емкость накопителя равна 0, то система \textit{с потерями}. Если емкость равна бесконечности~--- система \textit{с ожиданием}.

Для задания Q-схемы также необходимо описать алгоритм ее функционирования, который определяет набор правил поведения заявок в системе в различные моменты времени.

Неоднородность заявок, отражающая процесс в той или иной реальной системе, может быть определена классами приоритетов.

Весь набор возможных алгоритмов поведения заявок в Q-схемах можно представить в виде некоторого оператора алгоритма.
%
\begin{gather*}
    Q = (W, U, R, A, H, Z)
\end{gather*}
%
Здесь $W$~--- подмножество входящих потоков, $U$~--- подмножество потоков обслуживания, $R$~--- оператор сопряжения, $A$~--- оператор алгоритма, $H$~--- вектор внутренних параметров, $Z$~--- множество состояний.
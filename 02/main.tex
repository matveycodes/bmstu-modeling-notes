\section{Лекция №2 (18.09.2023)}

\subsection{Виды имитационного моделирования}

\subsubsection{Агентное моделирование}

Используется для исследования децентрализованных систем, динамика функционирования которых определяется не глобальными правилами и законами, а, наоборот, когда эти глобальные правила и законы являются результатом индивидуальной активности членов группы.

Цель агентной модели~--- получить представление об этих глобальных правилах, общем поведении системы исходя из предположений об индивидуальном, частном поведении отдельных активных объектов и взаимодействии этих объектов в системе.

Таким образом, объект~--- это некая сущность, обладающая активностью и способная принимать решения в соответствии с некоторым набором правил, взаимодействовать с окружающей средой и самостоятельно изменяться.

Пример: автомобильный трафик (объект~--- движущееся средство).

\subsubsection{Дискретно-событийное моделирование}

Предлагается абстрагироваться от непрерывной природы объекта и событий, присущих данным объектам. Рассматриваются только основные события: например, ожидание или обработка заказа. Данное направление наиболее развито приложениями.

Пример: производственные процессы, логистика.

\subsubsection{Системная динамика}

Для исследования системы строятся графические диаграммы причинных связей и глобальных влияний одних параметров объектов на другие. Основным параметром является время.

Выделяют три основных этапа в развитии средств имитационного моделирования.

\begin{enumerate}
    \item Создание имитационной модели на универсальном языке программирования. Могут использоваться специализированные языки компьютерного моделирования (например, GPSS), объектно-ориентированные языки моделирования.
    \item Использование при разработке имитационных моделей проб\-лем\-но-ориен\-тиро\-ван\-ных систем. Эти системы, как правило, не требуют знания программирования.
    \item Использование методов искусственного интеллекта.
\end{enumerate}

Самый сложный этап имитационного моделирования~--- формализация модели. Необходимо четко определить входные данные, внутреннее содержание модели, выходные данные и воздействия внешней среды.

Нерешенной проблемой остается присутствие в процессе имитационного моделирования \textbf{неформального} этапа перевода поставленной задачи на язык математики. Этот этап не может быть решен ни программистом, ни аналитиком самостоятельно~--- необходима работа нескольких человек разной квалификации.

\subsection{Технические средства математического моделирования}

\begin{itemize}
    \item как средства расчета по полученным аналитическим моделям;
    \item как средства разработки программ.
\end{itemize}

При моделировании используется два вида вычислительной техники.

\begin{enumerate}
    \item Цифровые машины:
          \begin{itemize}
              \item процессор;
              \item память;
              \item операционная система.
          \end{itemize}
    \item Аналоговые машины.
\end{enumerate}

В отличие от цифровой, в основе аналоговой вычислительной техники заложен принцип моделирования, а не счета. Каждой переменной величине задачи ставится в соответствие определенная переменная величина электронной цепи. При этом основой построения такой модели является изоморфизм~--- подобие исследуемой задачи и соответствующей ей электронной цепи.

\begin{dd}
    Аналоговая вычислительная машина~--- это совокупность электрических элементов, организованных в систему, позволяющую изоморфно моделировать динамику исследуемого объекта.
\end{dd}

\image
{\textwidth}
{02/inc/analog}
{Структурная схема АВМ}
\section{Лекция №4 (02.10.2023)}

\subsection{Базовые понятия теории систем}

\begin{dd}
    Система~--- множество элементов, находящихся в отношениях и связанных между собой.
\end{dd}

\begin{dd}
    Элемент~--- часть системы, представление о которой нецелесообразно подвергать дальнейшему членению.
\end{dd}

\begin{dd}
    Сложная система~--- система, характеризующаяся большим числом элементов и (что наиболее важно) большим числом взаимосвязей элементов. Сложная система определяется также видом взаимосвязей элементов, свойствами целенаправленности, целостности, членимости, иерархичности, многоаспектности.
\end{dd}

\begin{dd}
    Подсистема~--- часть системы (подмножество элементов и их взаимосвязей), которая имеет свойства системы.
\end{dd}

\begin{dd}
    Надсистема~--- система, по отношению к которой рассматриваемая система является подсистемой.
\end{dd}

\begin{dd}
    Структура~--- отображение совокупности элементов системы и их взаимосвязей.
\end{dd}

Понятие структуры отличается от понятия самой системы также еще и тем, что при описании структуры принимают во внимание лишь типы элементов и связей без конкретизации значений их параметров.

\begin{dd}
    Параметр~--- величина, выражающая свойства системы или ее части.
\end{dd}

\subsection{Характеристики сложных систем}

К главным относятся следующие характеристики сложных систем.

\begin{enumerate}
    \item Целенаправленность~--- свойство искусственной системы, выражающее назначение системы. Необходимо для оценки вариантов системы.
    \item Целостность~--- свойство системы, характеризующее взаимосвязанность элементов и наличие зависимости выходных параметров от параметров элементов, причем большинство выходных параметров не являются простым повторением или суммой параметров элементов.
    \item Иерархичность~--- свойство сложной системы, выражающее возможность и целесообразность ее иерархического описания, то есть представления в виде нескольких уровней, между компонентами которых имеется отношение <<часть~--- целое>>.
\end{enumerate}

Моделирование сложных дискретных систем имеет две задачи:

\begin{itemize}
    \item создание моделей (modeling);
    \item анализ свойств систем на основе исследования их моделей (simulation).
\end{itemize}

\subsection{Последовательность разработки и компьютерной реализации}

Сущность компьютерного моделирования систем состоит в проведении эксперимента с моделью~--- программой, описывающей формально или алгоритмически поведение элементов системы в процессе функционирования системы, то есть в их взаимодействии друг с другом и внешней средой.

Основные требования, предъявляемые к модели:

\begin{itemize}
    \item полнота модели должна предоставлять пользователю возможность получения необходимого набора характеристик, оценок системы;
    \item гибкость модели должна давать возможность воспроизведения различных ситуаций при варьировании структуры, алгоритмов и параметров моделей, причем структура должна быть блочной, то есть допускать возможность замены, добавления, исключения некоторых частей без изменения всей модели;
    \item компьютерная реализация модели должна соответствовать имеющимся техническим ресурсам.
\end{itemize}

Необходимость определяет оценку системы с требуемой точностью и достоверностью.

Процесс моделирования включает разработку и компьютерную реализацию модели. Является итерационным. Этот итерационный процесс продолжается до тех пор, пока не будет получена модель, которую можно считать адекватной в рамках решения поставленной задачи.

\subsection{Основные этапы моделирования больших систем}

\subsubsection{Первый этап}

На первом этапе построения концептуальной (описательной) модели системы формулируется модель, и строится ее формальная схема. Основным назначением данного этапа является переход от содержательного описания объекта к его математической модели. Это наиболее ответственный и наименее формализованный этап. Исходный материал~--- содержательное описание объекта.

\begin{enumerate}
    \item Проведение границы между системой и внешней средой.
    \item Исследование моделируемого объекта с точки зрения выделения основных составляющих процесса функционирования по отношению к цели моделирования.
    \item Переход от содержательного описания системы к формализованному, который в данной интерпретации сводится к исключению из рассмотрения некоторых второстепенных элементов описания, не оказывающих существенного влияния на ход процессов, исследуемых с помощью модели.
    \item Оставшиеся элементы модели (системы) группируются в следующие блоки:
          \begin{itemize}
              \item первой группы~--- имитатор воздействия внешней среды;
              \item второй группы~--- имитатор воздействия модели;
              \item третьей группы~--- качественная обработка результатов.
          \end{itemize}
    \item Процесс функционирования сложной системы разбивается на подпроцессы так, чтобы построение модели отдельных процессов было элементарно и не вызывало особых трудностей.
\end{enumerate}

\subsubsection{Второй этап}

На втором этапе~--- алгоритмизация модели (модель реализуется в виде программы).

\begin{enumerate}
    \item Разработка схемы моделирующего алгоритма.
    \item Разработка схемы программы (модели).
    \item Выбор технических средств для реализации программной модели.
    \item Программирование, отладка.
    \item Проверка достоверности на тестовых примерах.
    \item Составление технической документации (логические схемы, текст программы, спецификации).
\end{enumerate}

\subsubsection{Третий этап}

Третий этап моделирования~--- этап получения и интерпретации результатов.

\begin{enumerate}
    \item Планирование компьютерного эксперимента. Составление плана эксперимента с указанием комбинаций переменных, параметров. Главная задача~--- дать максимальный объем информации об объекте при минимальных затратах компьютерного времени.
    \item Проведение рабочих расчетов (контрольная калибровка модели).
    \item Статистическая обработка результатов эксперимента.
    \item Составление технической документации.
\end{enumerate}

\subsection{Классы ошибок}

Выделяют три класса ошибок.

\begin{enumerate}
    \item Ошибки формализации: недостаточно полная модель.
    \item Ошибки решения: метод не позволяет достичь заданной точности.
    \item Ошибки задания параметров модели.
\end{enumerate}

\image
{\textwidth}
{04/inc/scheme}
{Схема взаимосвязи технологических этапов моделирования}